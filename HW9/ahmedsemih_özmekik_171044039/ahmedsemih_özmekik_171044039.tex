\documentclass[12pt]{article}
\usepackage[utf8]{inputenc}
\usepackage{ragged2e}

\title{BIL101 INTRODUCTION TO COMPUTER SCIENCE}
\author{#HW09}
\date{Ahmed Semih Özmekik}

\begin{document}

\maketitle


\centering
\section{Pekiştirmeli Öğrenme}
Makine öğrenmesi yöntemlerinden birisi olan pekiştirmeli öğrenmeyi ,
	türkçedeki karşılığının çağrışımlarından yola çıkarak
	tanımaya çalışmak çok yanlış olmaz.Çünkü gerçek anlamda deneme yanılma yani "pekiştirerek"
	spesifik amacına ulaşmaya çalışan bir öğrenme çeşidi olduğunu söylemek mümkün.
	Pekiştirmeli öğrenmeyi diğer makine öğrenmesi yöntemlerinden ayıran özellikler
	dünyamızdaki canlıların öğrenme yeteneklerinden çok da farklı olmayışı ve amaca
	odaklı bir yöntem olması.Örneğin biyolojide solucanların öğrenme yeteneklerinin
	test edildiği
	bir düzenekte , solucanlara 2 farklı yol sunuluyor ve 1.yolu seçtikleri takdirde
	küçük bir elektrik çarpması ile karşılıyorlar , 2.yolu seçtikleri takdirde
	besin ile karşılaşıyorlar.Ve testin ilerleyen aşamalarında solucanların artık 1.yolu
	seçmedikleri gözleniyor.İşte pekiştirmeli öğrenme yöntemini bu örnekteki
	öğrenme yöntemiyle bağdaştırabilmemiz mümkün.Bu benzerliği pekiştirmeli öğrenme
	yönteminin detaylarına girerek daha iyi netleştirmeye çalışalım. Yöntemde
	Bir tane öğrenen etken(agent) ve öğretici çevre (enviroment) bulunmakta.Amacımız
	öğrenen'i çevre ile etkileşime sokarak bu etkileşimlerin sonucunda çevreden
	aldığımız geribildirimle ögrenen'i sokmak istediğimiz ideal kalıba sokuyoruz.
	Bu geribildirimlerin pozitif ya da negatif olabilmesi öğrenen'imizi işte bu şekilde
	eğitiyoruz.Görüldüğü üzere elimizde spesifik bir amaç yani bir ideal model olmak zorunda.
	Bu bağlamda pekiştirmeli öğrenme yöntemi tıpkı solucanların 1.yolu seçmesinin ardından
	canlarının yanmasıyla bir daha 1.yolu seçmediklerindeki kullandıkları methoda benziyor.


\section{Görüntü İşleme ve Grafik Teknikleri}
Görüntü işlemeyi , var olan hareketli ya da hareketsiz görüntüyü
dijital ortama aktarma işlemi olarak tanımlayabiliriz.Bunun yanında detaylandırmak
istersek , girdisi görüntü ve görüntü türevleri olan , çıktısı yine görüntü ya da
görüntü ile alakalı karakteristik yararlı bilgiler olabilen bir tür sinyal işlemedir.
Bugünlerde hızla büyüyen ve oldukça popüler bir araştırma konusu olan
görüntü işlemi 3 basit adımda ele alabiliriz.Bir takım araçlarla görüntü girdi
olarak aldığımız adım.Görütünün analiz edildiğini ve işlenildiği adım.Ve sonucumuzun
istediğimiz formatta çıktı olarak verildiği adım.
Bunun yanında 2 boyutlu ve 3 boyutlu grafik işlemenin detayları ve farklarından
bahsedecek olursak ise konuyu film şeridi üzerinde ele almak gerekir.Buradaki
film şeridinin uygulaması ise aslında işlemenin 2 boyutlu mu yoksa 3 boyutlu mu
değişir.2 boyutlu işlemenin ele alındığı bir projede , film şeridi
bir çok küçük karelere (keyframe) ayrılır bu şekilde düzenlenir.Bu düzenlenmekte
olan her sahne için detaylı bir uygulama sağlar.Şimdilerde ise bu yöntemin alternatifi
olarak görüntü işleme ve 2 boyutlu yazılımlar sayesinde sahne araları otomatik olarak
doldurulur ve her sahnenin ayrı ayrı ele alınma zahmetinden kurtulmuş olurlar.3 boyutlu
görüntü işlemede ise birinci aşama dinamikler yani bildiğimiz gerçek anlamları ile
fizik yasalarının bu görsellere implementasyonu gerçekleştirilir.İkinci adımında ise
kinematikler yani cisimlerin birbirlerine bağlı olarak hareketlerini simule edip
uygulamaya çalışıldığı adımdır.Son adım ise hareket yakalama işlemiyle spesifik model üzerinde
uygulanmak istenen hareketin o modelin hareketlerini inceleyerek kayıt altına alınması
sonrasında gerçekleştirilmesi.Örneğin insan üzerine yapıştırılan noktalarla hareketin incelenmesi
ve sonra implementasyonun gerçekleştirilmesi gibi.
\end{document}
